\documentclass[12pt]{article}
\usepackage{dsfont}
\usepackage{cite}
\usepackage{amsmath}
\usepackage{amsfonts}
\usepackage{mathbbol}
\usepackage{graphicx}
\usepackage{float}
\usepackage{siunitx}
\usepackage{hyperref}
\usepackage{listings}
\usepackage[cm]{fullpage}
\usepackage{layout}
\begin{document}
\title{Useful Equations}
\author{Emily Cranwell and Daniel Huantes}
\maketitle
\section{Guidelines}
Dump equations here with titles and with a brief blurb about what it is, assumptions it makes, etc
\section{Heat Transfer Equations}
\subsection{Fourier's law of heat conduction}
\begin{equation}
\begin{aligned}
\boldsymbol{q} = -k\boldsymbol{\nabla} T
\end{aligned}
\label{fig:FLHC}
\end{equation} 
$\boldsymbol{q}$ is the local heat flux density $\si{\watt\meter^2}$,
$k$ is the material's conductivity$\si{\watt\meter^1\kelvin^-1}$, and in an anisotropic medium, k is a second order tensor that varies with location if the material is non-uniform
$T$ is the temperature $\si{\kelvin\meter^-1}$
\subsection{Thermal Current Conservation}
\begin{equation}
\begin{aligned}
\nabla\cdot\boldsymbol{q}(\boldsymbol{x}, t) + \frac{k}{\alpha} \frac{\partial T}{\partial t}(\boldsymbol{x}, t) = S(\boldsymbol{x}, t)
\label{fig:TCC}
\end{aligned}
\end{equation}
$\boldsymbol{q}$ is the local heat flux density, $\alpha$ is 
\begin{math}
k = \frac{k}{\rho c}
\end{math}
where $c$ is the specific heat, and $\rho$ is the density
\subsection{Parabolic Heat Transfer Equation}
\begin{equation}
\begin{aligned}
\nabla\cdot\boldsymbol{q}(\boldsymbol{x}, t) + \frac{k}{\alpha} \frac{\partial T}{\partial t}(\boldsymbol{x}, t) = S(\boldsymbol{x}, t)
\label{fig:PHTE}
\end{aligned}
\end{equation}
Assuming thermal current conservation and Fourier's law hold directly yieldsthis
\subsection{Pennes Bioheat Equation}
\begin{equation}
\begin{aligned}
S(\boldsymbol{x}, t) = S_s(\boldsymbol{x}, t) + S_p(\boldsymbol{x}, t) + S_m(\boldsymbol{x}, t)
\label{fig:PBHE}
\end{aligned}
\end{equation}





\end{document}
