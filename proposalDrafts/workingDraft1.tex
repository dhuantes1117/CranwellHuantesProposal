\documentclass[12pt]{article}
\usepackage{dsfont}
\usepackage{cite}
\usepackage{amsmath}
\usepackage{amsfonts}
\usepackage{mathbbol}
\usepackage{graphicx}
\usepackage{float}
\usepackage{siunitx}
\usepackage{hyperref}
\usepackage{listings}
\usepackage[cm]{fullpage}
\usepackage{layout}
\begin{document}
\title{Research Proposal
\large draft 1}
\author{Emily Cranwell and Daniel Huantes}
\maketitle
\section{Introduction}
Thermal conduction in tissue is firmly within TSRL's main area of interest: laser bio effects. Whenever lasers come into contact with tissue, unique problems arise that makes computational modeling extremely valuable in determining a variety of effects. The research through Fort Hays with SAIC that Dan and Emily have worked on has dealt with focusing on different parts of thermal modeling. Between the two of us we have worked on a tool to generate linear combinations of temperature profiles, and dealt with angular transformations of the heat equation with irregular spacing conditions. We believe these skills are applicable to a research project based on modeling thermal effects in tissue, including potential applications for the use of damage threshold calculations.

\section{Motivation}
Our motivation for this is driven by the vast amounts of physical applications of laser bio effects, especially in medical practices. Models are useful as physical experiments are often difficult to design and carryout due to the danger presented in using human subjects and difficulty in finding accurate substitutes. Each situation will require extensive knowledge of physical and biological components, which may vary greatly. This shows the usefulness of a computational model that can provide flexibility in such biological and physical areas. A better understanding of heat transfer is also useful in determining damage threshold calculations for specific configurations. 

\section{Why is this physically difficult}

\subsection{Biology/BioPhysics}
Heat transfer is affected by vessel geometry, blood flow rates (which changes locally since arteries and veins have different flow rates), and the thermal capacity of blood. We know that blood enters through arteries and flows to the tissue cells through blood capillaries, and this movement affects heat flow. 

The bioheat tranfer equation takes into account both the convective flow effects of blood perfusion and heat conduction. The Pennes 1D bioheat transfer equation gives satisfactory results when the direction of heat propagation is in a perpendicular direction to the skin surface, so such a geometry would allow for simplification in modeling. 
However, Pennes made quite a few assumpations that have since been proven wrong, an example being vessel size. The bioheat equation does not account for large blood vessels, for example. 
	
\subsection{Computationally/mathematically}
Modeling transient heat transfer in tissue presents several unique challenges, making finding an analytic solution to problems to be impossible. The difficulties of this problem lie in 3 parts: the geometries, the unknown properties, and modeling transient diffusive heat transfer.
Tissue is weakly homogenous, meaning it's properies vary within the medium based on position, but stay reasonably consistent. The physical composition of a material greatly influences heat transfer as well, and a computational model should use a geometry reflecting this.  

The next issue is finding values for the physical properties of tissue in question. This could be done by characterizing the physical properties of some tissue-like thermal phantom, or through accessing recorded values. The isotropy of the material comes into play as well, as due to the weak homogeneity of the material, the thermal conductivity, $k$ can vary slightly with position. In perfectly isotropic media, $k$ is a constant scalar, while in a material that exhibits anisitropy, or in which heat may diffuse more readily in one direction than another, $k$ is a rank 2 tensor. 

Finally, the temperature throughout the medium is location and time dependent, and while regions of relatively similar properies can be identified, solving the heat equation at each value of $t$ in each region subject to boundary conditions that change through each time step could present a large computational strain. This issue's main source is that for relatively long timescales, the transient response may have large effects, especially in the case of situations in which the source dissapears before a steady state solution is reached, like for a pulsed laser. 

\section{Methodology}

Our plan is to create a 3D time-dependent thermal solver to measure heat flow in weakly homogenous media in C++, using existing libraries that Dr. Clark has developed. By applying a hyperbolic heat equation in conjunction with numerical techniques to deal with the unique geometry that weakly homogenous tissue presents. By working with the libraries Dr. Clark has created we save time in terms of developing frameworks for modeling 3D geometries, as well as having any work we may accomplish easily able to be furthered, such as through the use of LibArrhenius to calculate damage thresholds. In terms of the long term we also have an idea for using grocery store meat in order to have a relatively cheap, replicable thing to qualitatively compare

\subsection{Equation ideas}
\subsection{Numerical Methods/Solvers}
\section{Front propagation}
\section{Goals}

\end{document}
